\documentclass{article}
\usepackage[utf8]{inputenc}
\usepackage{pgfplotstable}
\pgfplotsset{compat=1.15}
\usepackage{fancyvrb}

\title{}
\author{}
\date{}

\begin{document}

\maketitle

\section{Introduction}

Using our DenseNet-BC we have approximately $634k$ 32FP-ops and $626k$ FP32-params. This gives us a score of approximately (rounding up) $0.078$ in total, where for the ops, we obtain $0.06$ as our score, and for the parameters, we obtain $0.018$ as our score.

To test a pre-trained network and check the score, just run profile.py. To train a network, just run train.py and then re-run profile.py.  

We describe our methods and computations in the following:

\section{Data Augmentation}
\label{data_aug}

We first use standard Data augmentation such as flip and random crop. In addition, we propose to add other different data augmentations. We use CutMix with $\alpha = 1$ for all training dataset, and we drop $64$ pixels randomly for each input training image. 

\section{Architecture}

We use a DenseNet-BC architecture with $K = 8$, depth $D = 196$ and reduction $R = 0.5$. 

\section{Computing the Number of Parameters}

To get the number of parameters of the DenseNet-BC architecture, we compute the number of parameters of each layer (or each Bottleneck layer that contains 2 Batch norm layers, 2 Convolutional layers and 2 Relu), as described in Table~\ref{table:params}

\begin{table}[h]
    \centering
    \begin{tabular}{c|c}
    Layer     & Number of parameters  \\
    \hline
    Convolution     &      $F_{in}F_{out}S_c^2$ \\
    \hline
    Batch norm & $4F_{in}$ \\
    \hline
    $Bottleneck_i$ & $ 4[F_{in} + K(4 + F_{in} + i + K(1 + i)]$\\
    \hline
    Linear & $F_{in}F_{out}$\\
    \hline
    Transition & $F_{in}(4 + RF_{in})$ \\
    \hline
    \end{tabular}
    \caption{The number of parameters for each layer. Note that $F_{in}$ represents the number of input feature maps, $F_{out}$ represents the number of output feature maps, $S_c^2$ represents the kernel size of a convolutional operation, $i$ represents the depth of the bottleneck inside a block that contains $\frac{D-4}{6}$ bottleneck layers.}
    \label{table:params}
\end{table}

\section{Computing the Number of Operations}

to get the number of operations required to process one input through the  DenseNet-BC network, we compute the number of operations of each layer (or each Bottleneck layer that contains 2 Batch norm layers, 2 Convolutional layers and 2 Relu), as described in Table~\ref{table:Flops}


\begin{table}[h]
    \centering
    \begin{tabular}{c|c}
    Layer     & Number of operations  \\
    \hline
    Convolution     &  $[\frac{F_{in}F_{out}S_c^2}{2} + (F_{in}S_c^2 -1)F_{out}]P $ \\
    \hline
    Batch norm & $3F_{in}P$ \\
    \hline
    Relu & $F_{in}P$\\
    \hline
    Average pooling & $S_p^2PF_{in} $ \\
    \hline
    Linear & $1.5F_{in}F_{out}$ \\
    \hline
    Transition & $4F_{in}P + [\frac{F_{in}F_{out}S_c^2}{2} + (F_{in}S_c^2 -1)F_{out}]P + S_p^2PF_{out}$ \\
    \hline
    $Bottleneck_i$ & Batch norm1 + Relu1 + Batch norm2 + Relu2 + Conv1 + Conv2\\
    & Batch norm1 = $3P(F_{in} + iK)$\\
    & Relu1 = $P(F_{in} + iK)$\\
    & Batch norm2 = $12P_{in}K$\\
    & Relu2 = $4P_{in}K$\\
    & Conv1 = $P[\frac{(F_{in} + iK)4K}{2} + (F_{in} + ik -1)4K]$\\
    & Conv2 = $[\frac{36K^2}{2} + (36K-1)K]P$\\
    \hline
    
    \hline
    
    \hline
    \end{tabular}
    \caption{The number of operations for each layer. Note that $F_{in}$ represents the number of input feature maps, $F_{out}$ represents the number of output feature maps, $S_c^2$ represents the kernel size of a convolutional operation, $S_p^2$ represents the kernel size of an average pooling operation, $i$ represents the depth of the bottleneck inside a block that contains $\frac{D-4}{6}$ bottleneck layers, $P$ represents the number of pixels of an input feature map.}
    \label{table:Flops}
\end{table}

\section{Training procedure}

We train the network with the data augmentation described on Section~\ref{data_aug} for 200 epochs, using a batch size of 32 and a softmax temperature of 6. Our learning rate starts at 0.1 and is divided by 10 at epochs 100 and 150. We then fine-tune the network by training it for an extra 5 epochs without data-augmentation. 

\newpage

\begin{Verbatim}[fontsize=\small]
Conv2d: S_c=3, F_in=3, F_out=16, P=1024, params=216, operations=647168
Batch norm: F_in=16 P=1024, params=16, operations=49152
ReLU: F_in=16 P=1024, params=0, operations=16384
Conv2d: S_c=1, F_in=16, F_out=32, P=1024, params=256, operations=753664
Batch norm: F_in=32 P=1024, params=32, operations=98304
ReLU: F_in=32 P=1024, params=0, operations=32768
Conv2d: S_c=3, F_in=32, F_out=8, P=1024, params=1152, operations=3530752
Batch norm: F_in=24 P=1024, params=24, operations=73728
ReLU: F_in=24 P=1024, params=0, operations=24576
Conv2d: S_c=1, F_in=24, F_out=32, P=1024, params=384, operations=1146880
Batch norm: F_in=32 P=1024, params=32, operations=98304
ReLU: F_in=32 P=1024, params=0, operations=32768
Conv2d: S_c=3, F_in=32, F_out=8, P=1024, params=1152, operations=3530752
Batch norm: F_in=32 P=1024, params=32, operations=98304
ReLU: F_in=32 P=1024, params=0, operations=32768
Conv2d: S_c=1, F_in=32, F_out=32, P=1024, params=512, operations=1540096
Batch norm: F_in=32 P=1024, params=32, operations=98304
ReLU: F_in=32 P=1024, params=0, operations=32768
Conv2d: S_c=3, F_in=32, F_out=8, P=1024, params=1152, operations=3530752
Batch norm: F_in=40 P=1024, params=40, operations=122880
ReLU: F_in=40 P=1024, params=0, operations=40960
Conv2d: S_c=1, F_in=40, F_out=32, P=1024, params=640, operations=1933312
Batch norm: F_in=32 P=1024, params=32, operations=98304
ReLU: F_in=32 P=1024, params=0, operations=32768
Conv2d: S_c=3, F_in=32, F_out=8, P=1024, params=1152, operations=3530752
Batch norm: F_in=48 P=1024, params=48, operations=147456
ReLU: F_in=48 P=1024, params=0, operations=49152
Conv2d: S_c=1, F_in=48, F_out=32, P=1024, params=768, operations=2326528
Batch norm: F_in=32 P=1024, params=32, operations=98304
ReLU: F_in=32 P=1024, params=0, operations=32768
Conv2d: S_c=3, F_in=32, F_out=8, P=1024, params=1152, operations=3530752
Batch norm: F_in=56 P=1024, params=56, operations=172032
ReLU: F_in=56 P=1024, params=0, operations=57344
Conv2d: S_c=1, F_in=56, F_out=32, P=1024, params=896, operations=2719744
Batch norm: F_in=32 P=1024, params=32, operations=98304
ReLU: F_in=32 P=1024, params=0, operations=32768
Conv2d: S_c=3, F_in=32, F_out=8, P=1024, params=1152, operations=3530752
Batch norm: F_in=64 P=1024, params=64, operations=196608
ReLU: F_in=64 P=1024, params=0, operations=65536
Conv2d: S_c=1, F_in=64, F_out=32, P=1024, params=1024, operations=3112960
Batch norm: F_in=32 P=1024, params=32, operations=98304
ReLU: F_in=32 P=1024, params=0, operations=32768
Conv2d: S_c=3, F_in=32, F_out=8, P=1024, params=1152, operations=3530752
Batch norm: F_in=72 P=1024, params=72, operations=221184
ReLU: F_in=72 P=1024, params=0, operations=73728
Conv2d: S_c=1, F_in=72, F_out=32, P=1024, params=1152, operations=3506176
Batch norm: F_in=32 P=1024, params=32, operations=98304
ReLU: F_in=32 P=1024, params=0, operations=32768
Conv2d: S_c=3, F_in=32, F_out=8, P=1024, params=1152, operations=3530752
Batch norm: F_in=80 P=1024, params=80, operations=245760
ReLU: F_in=80 P=1024, params=0, operations=81920
Conv2d: S_c=1, F_in=80, F_out=32, P=1024, params=1280, operations=3899392
Batch norm: F_in=32 P=1024, params=32, operations=98304
ReLU: F_in=32 P=1024, params=0, operations=32768
Conv2d: S_c=3, F_in=32, F_out=8, P=1024, params=1152, operations=3530752
Batch norm: F_in=88 P=1024, params=88, operations=270336
ReLU: F_in=88 P=1024, params=0, operations=90112
Conv2d: S_c=1, F_in=88, F_out=32, P=1024, params=1408, operations=4292608
Batch norm: F_in=32 P=1024, params=32, operations=98304
ReLU: F_in=32 P=1024, params=0, operations=32768
Conv2d: S_c=3, F_in=32, F_out=8, P=1024, params=1152, operations=3530752
Batch norm: F_in=96 P=1024, params=96, operations=294912
ReLU: F_in=96 P=1024, params=0, operations=98304
Conv2d: S_c=1, F_in=96, F_out=32, P=1024, params=1536, operations=4685824
Batch norm: F_in=32 P=1024, params=32, operations=98304
ReLU: F_in=32 P=1024, params=0, operations=32768
Conv2d: S_c=3, F_in=32, F_out=8, P=1024, params=1152, operations=3530752
Batch norm: F_in=104 P=1024, params=104, operations=319488
ReLU: F_in=104 P=1024, params=0, operations=106496
Conv2d: S_c=1, F_in=104, F_out=32, P=1024, params=1664, operations=5079040
Batch norm: F_in=32 P=1024, params=32, operations=98304
ReLU: F_in=32 P=1024, params=0, operations=32768
Conv2d: S_c=3, F_in=32, F_out=8, P=1024, params=1152, operations=3530752
Batch norm: F_in=112 P=1024, params=112, operations=344064
ReLU: F_in=112 P=1024, params=0, operations=114688
Conv2d: S_c=1, F_in=112, F_out=32, P=1024, params=1792, operations=5472256
Batch norm: F_in=32 P=1024, params=32, operations=98304
ReLU: F_in=32 P=1024, params=0, operations=32768
Conv2d: S_c=3, F_in=32, F_out=8, P=1024, params=1152, operations=3530752
Batch norm: F_in=120 P=1024, params=120, operations=368640
ReLU: F_in=120 P=1024, params=0, operations=122880
Conv2d: S_c=1, F_in=120, F_out=32, P=1024, params=1920, operations=5865472
Batch norm: F_in=32 P=1024, params=32, operations=98304
ReLU: F_in=32 P=1024, params=0, operations=32768
Conv2d: S_c=3, F_in=32, F_out=8, P=1024, params=1152, operations=3530752
Batch norm: F_in=128 P=1024, params=128, operations=393216
ReLU: F_in=128 P=1024, params=0, operations=131072
Conv2d: S_c=1, F_in=128, F_out=32, P=1024, params=2048, operations=6258688
Batch norm: F_in=32 P=1024, params=32, operations=98304
ReLU: F_in=32 P=1024, params=0, operations=32768
Conv2d: S_c=3, F_in=32, F_out=8, P=1024, params=1152, operations=3530752
Batch norm: F_in=136 P=1024, params=136, operations=417792
ReLU: F_in=136 P=1024, params=0, operations=139264
Conv2d: S_c=1, F_in=136, F_out=32, P=1024, params=2176, operations=6651904
Batch norm: F_in=32 P=1024, params=32, operations=98304
ReLU: F_in=32 P=1024, params=0, operations=32768
Conv2d: S_c=3, F_in=32, F_out=8, P=1024, params=1152, operations=3530752
Batch norm: F_in=144 P=1024, params=144, operations=442368
ReLU: F_in=144 P=1024, params=0, operations=147456
Conv2d: S_c=1, F_in=144, F_out=32, P=1024, params=2304, operations=7045120
Batch norm: F_in=32 P=1024, params=32, operations=98304
ReLU: F_in=32 P=1024, params=0, operations=32768
Conv2d: S_c=3, F_in=32, F_out=8, P=1024, params=1152, operations=3530752
Batch norm: F_in=152 P=1024, params=152, operations=466944
ReLU: F_in=152 P=1024, params=0, operations=155648
Conv2d: S_c=1, F_in=152, F_out=32, P=1024, params=2432, operations=7438336
Batch norm: F_in=32 P=1024, params=32, operations=98304
ReLU: F_in=32 P=1024, params=0, operations=32768
Conv2d: S_c=3, F_in=32, F_out=8, P=1024, params=1152, operations=3530752
Batch norm: F_in=160 P=1024, params=160, operations=491520
ReLU: F_in=160 P=1024, params=0, operations=163840
Conv2d: S_c=1, F_in=160, F_out=32, P=1024, params=2560, operations=7831552
Batch norm: F_in=32 P=1024, params=32, operations=98304
ReLU: F_in=32 P=1024, params=0, operations=32768
Conv2d: S_c=3, F_in=32, F_out=8, P=1024, params=1152, operations=3530752
Batch norm: F_in=168 P=1024, params=168, operations=516096
ReLU: F_in=168 P=1024, params=0, operations=172032
Conv2d: S_c=1, F_in=168, F_out=32, P=1024, params=2688, operations=8224768
Batch norm: F_in=32 P=1024, params=32, operations=98304
ReLU: F_in=32 P=1024, params=0, operations=32768
Conv2d: S_c=3, F_in=32, F_out=8, P=1024, params=1152, operations=3530752
Batch norm: F_in=176 P=1024, params=176, operations=540672
ReLU: F_in=176 P=1024, params=0, operations=180224
Conv2d: S_c=1, F_in=176, F_out=32, P=1024, params=2816, operations=8617984
Batch norm: F_in=32 P=1024, params=32, operations=98304
ReLU: F_in=32 P=1024, params=0, operations=32768
Conv2d: S_c=3, F_in=32, F_out=8, P=1024, params=1152, operations=3530752
Batch norm: F_in=184 P=1024, params=184, operations=565248
ReLU: F_in=184 P=1024, params=0, operations=188416
Conv2d: S_c=1, F_in=184, F_out=32, P=1024, params=2944, operations=9011200
Batch norm: F_in=32 P=1024, params=32, operations=98304
ReLU: F_in=32 P=1024, params=0, operations=32768
Conv2d: S_c=3, F_in=32, F_out=8, P=1024, params=1152, operations=3530752
Batch norm: F_in=192 P=1024, params=192, operations=589824
ReLU: F_in=192 P=1024, params=0, operations=196608
Conv2d: S_c=1, F_in=192, F_out=32, P=1024, params=3072, operations=9404416
Batch norm: F_in=32 P=1024, params=32, operations=98304
ReLU: F_in=32 P=1024, params=0, operations=32768
Conv2d: S_c=3, F_in=32, F_out=8, P=1024, params=1152, operations=3530752
Batch norm: F_in=200 P=1024, params=200, operations=614400
ReLU: F_in=200 P=1024, params=0, operations=204800
Conv2d: S_c=1, F_in=200, F_out=32, P=1024, params=3200, operations=9797632
Batch norm: F_in=32 P=1024, params=32, operations=98304
ReLU: F_in=32 P=1024, params=0, operations=32768
Conv2d: S_c=3, F_in=32, F_out=8, P=1024, params=1152, operations=3530752
Batch norm: F_in=208 P=1024, params=208, operations=638976
ReLU: F_in=208 P=1024, params=0, operations=212992
Conv2d: S_c=1, F_in=208, F_out=32, P=1024, params=3328, operations=10190848
Batch norm: F_in=32 P=1024, params=32, operations=98304
ReLU: F_in=32 P=1024, params=0, operations=32768
Conv2d: S_c=3, F_in=32, F_out=8, P=1024, params=1152, operations=3530752
Batch norm: F_in=216 P=1024, params=216, operations=663552
ReLU: F_in=216 P=1024, params=0, operations=221184
Conv2d: S_c=1, F_in=216, F_out=32, P=1024, params=3456, operations=10584064
Batch norm: F_in=32 P=1024, params=32, operations=98304
ReLU: F_in=32 P=1024, params=0, operations=32768
Conv2d: S_c=3, F_in=32, F_out=8, P=1024, params=1152, operations=3530752
Batch norm: F_in=224 P=1024, params=224, operations=688128
ReLU: F_in=224 P=1024, params=0, operations=229376
Conv2d: S_c=1, F_in=224, F_out=32, P=1024, params=3584, operations=10977280
Batch norm: F_in=32 P=1024, params=32, operations=98304
ReLU: F_in=32 P=1024, params=0, operations=32768
Conv2d: S_c=3, F_in=32, F_out=8, P=1024, params=1152, operations=3530752
Batch norm: F_in=232 P=1024, params=232, operations=712704
ReLU: F_in=232 P=1024, params=0, operations=237568
Conv2d: S_c=1, F_in=232, F_out=32, P=1024, params=3712, operations=11370496
Batch norm: F_in=32 P=1024, params=32, operations=98304
ReLU: F_in=32 P=1024, params=0, operations=32768
Conv2d: S_c=3, F_in=32, F_out=8, P=1024, params=1152, operations=3530752
Batch norm: F_in=240 P=1024, params=240, operations=737280
ReLU: F_in=240 P=1024, params=0, operations=245760
Conv2d: S_c=1, F_in=240, F_out=32, P=1024, params=3840, operations=11763712
Batch norm: F_in=32 P=1024, params=32, operations=98304
ReLU: F_in=32 P=1024, params=0, operations=32768
Conv2d: S_c=3, F_in=32, F_out=8, P=1024, params=1152, operations=3530752
Batch norm: F_in=248 P=1024, params=248, operations=761856
ReLU: F_in=248 P=1024, params=0, operations=253952
Conv2d: S_c=1, F_in=248, F_out=32, P=1024, params=3968, operations=12156928
Batch norm: F_in=32 P=1024, params=32, operations=98304
ReLU: F_in=32 P=1024, params=0, operations=32768
Conv2d: S_c=3, F_in=32, F_out=8, P=1024, params=1152, operations=3530752
Batch norm: F_in=256 P=1024, params=256, operations=786432
ReLU: F_in=256 P=1024, params=0, operations=262144
Conv2d: S_c=1, F_in=256, F_out=32, P=1024, params=4096, operations=12550144
Batch norm: F_in=32 P=1024, params=32, operations=98304
ReLU: F_in=32 P=1024, params=0, operations=32768
Conv2d: S_c=3, F_in=32, F_out=8, P=1024, params=1152, operations=3530752
Batch norm: F_in=264 P=1024, params=264, operations=811008
ReLU: F_in=264 P=1024, params=0, operations=270336
Conv2d: S_c=1, F_in=264, F_out=32, P=1024, params=4224, operations=12943360
Batch norm: F_in=32 P=1024, params=32, operations=98304
ReLU: F_in=32 P=1024, params=0, operations=32768
Conv2d: S_c=3, F_in=32, F_out=8, P=1024, params=1152, operations=3530752
Batch norm: F_in=272 P=1024, params=272, operations=835584
ReLU: F_in=272 P=1024, params=0, operations=278528
Conv2d: S_c=1, F_in=272, F_out=136, P=1024, params=18496, operations=56680448
AvgPool: S=2, F_in=136, P=1024, params=0, operations=69632
Batch norm: F_in=136 P=256, params=136, operations=104448
ReLU: F_in=136 P=256, params=0, operations=34816
Conv2d: S_c=1, F_in=136, F_out=32, P=256, params=2176, operations=1662976
Batch norm: F_in=32 P=256, params=32, operations=24576
ReLU: F_in=32 P=256, params=0, operations=8192
Conv2d: S_c=3, F_in=32, F_out=8, P=256, params=1152, operations=882688
Batch norm: F_in=144 P=256, params=144, operations=110592
ReLU: F_in=144 P=256, params=0, operations=36864
Conv2d: S_c=1, F_in=144, F_out=32, P=256, params=2304, operations=1761280
Batch norm: F_in=32 P=256, params=32, operations=24576
ReLU: F_in=32 P=256, params=0, operations=8192
Conv2d: S_c=3, F_in=32, F_out=8, P=256, params=1152, operations=882688
Batch norm: F_in=152 P=256, params=152, operations=116736
ReLU: F_in=152 P=256, params=0, operations=38912
Conv2d: S_c=1, F_in=152, F_out=32, P=256, params=2432, operations=1859584
Batch norm: F_in=32 P=256, params=32, operations=24576
ReLU: F_in=32 P=256, params=0, operations=8192
Conv2d: S_c=3, F_in=32, F_out=8, P=256, params=1152, operations=882688
Batch norm: F_in=160 P=256, params=160, operations=122880
ReLU: F_in=160 P=256, params=0, operations=40960
Conv2d: S_c=1, F_in=160, F_out=32, P=256, params=2560, operations=1957888
Batch norm: F_in=32 P=256, params=32, operations=24576
ReLU: F_in=32 P=256, params=0, operations=8192
Conv2d: S_c=3, F_in=32, F_out=8, P=256, params=1152, operations=882688
Batch norm: F_in=168 P=256, params=168, operations=129024
ReLU: F_in=168 P=256, params=0, operations=43008
Conv2d: S_c=1, F_in=168, F_out=32, P=256, params=2688, operations=2056192
Batch norm: F_in=32 P=256, params=32, operations=24576
ReLU: F_in=32 P=256, params=0, operations=8192
Conv2d: S_c=3, F_in=32, F_out=8, P=256, params=1152, operations=882688
Batch norm: F_in=176 P=256, params=176, operations=135168
ReLU: F_in=176 P=256, params=0, operations=45056
Conv2d: S_c=1, F_in=176, F_out=32, P=256, params=2816, operations=2154496
Batch norm: F_in=32 P=256, params=32, operations=24576
ReLU: F_in=32 P=256, params=0, operations=8192
Conv2d: S_c=3, F_in=32, F_out=8, P=256, params=1152, operations=882688
Batch norm: F_in=184 P=256, params=184, operations=141312
ReLU: F_in=184 P=256, params=0, operations=47104
Conv2d: S_c=1, F_in=184, F_out=32, P=256, params=2944, operations=2252800
Batch norm: F_in=32 P=256, params=32, operations=24576
ReLU: F_in=32 P=256, params=0, operations=8192
Conv2d: S_c=3, F_in=32, F_out=8, P=256, params=1152, operations=882688
Batch norm: F_in=192 P=256, params=192, operations=147456
ReLU: F_in=192 P=256, params=0, operations=49152
Conv2d: S_c=1, F_in=192, F_out=32, P=256, params=3072, operations=2351104
Batch norm: F_in=32 P=256, params=32, operations=24576
ReLU: F_in=32 P=256, params=0, operations=8192
Conv2d: S_c=3, F_in=32, F_out=8, P=256, params=1152, operations=882688
Batch norm: F_in=200 P=256, params=200, operations=153600
ReLU: F_in=200 P=256, params=0, operations=51200
Conv2d: S_c=1, F_in=200, F_out=32, P=256, params=3200, operations=2449408
Batch norm: F_in=32 P=256, params=32, operations=24576
ReLU: F_in=32 P=256, params=0, operations=8192
Conv2d: S_c=3, F_in=32, F_out=8, P=256, params=1152, operations=882688
Batch norm: F_in=208 P=256, params=208, operations=159744
ReLU: F_in=208 P=256, params=0, operations=53248
Conv2d: S_c=1, F_in=208, F_out=32, P=256, params=3328, operations=2547712
Batch norm: F_in=32 P=256, params=32, operations=24576
ReLU: F_in=32 P=256, params=0, operations=8192
Conv2d: S_c=3, F_in=32, F_out=8, P=256, params=1152, operations=882688
Batch norm: F_in=216 P=256, params=216, operations=165888
ReLU: F_in=216 P=256, params=0, operations=55296
Conv2d: S_c=1, F_in=216, F_out=32, P=256, params=3456, operations=2646016
Batch norm: F_in=32 P=256, params=32, operations=24576
ReLU: F_in=32 P=256, params=0, operations=8192
Conv2d: S_c=3, F_in=32, F_out=8, P=256, params=1152, operations=882688
Batch norm: F_in=224 P=256, params=224, operations=172032
ReLU: F_in=224 P=256, params=0, operations=57344
Conv2d: S_c=1, F_in=224, F_out=32, P=256, params=3584, operations=2744320
Batch norm: F_in=32 P=256, params=32, operations=24576
ReLU: F_in=32 P=256, params=0, operations=8192
Conv2d: S_c=3, F_in=32, F_out=8, P=256, params=1152, operations=882688
Batch norm: F_in=232 P=256, params=232, operations=178176
ReLU: F_in=232 P=256, params=0, operations=59392
Conv2d: S_c=1, F_in=232, F_out=32, P=256, params=3712, operations=2842624
Batch norm: F_in=32 P=256, params=32, operations=24576
ReLU: F_in=32 P=256, params=0, operations=8192
Conv2d: S_c=3, F_in=32, F_out=8, P=256, params=1152, operations=882688
Batch norm: F_in=240 P=256, params=240, operations=184320
ReLU: F_in=240 P=256, params=0, operations=61440
Conv2d: S_c=1, F_in=240, F_out=32, P=256, params=3840, operations=2940928
Batch norm: F_in=32 P=256, params=32, operations=24576
ReLU: F_in=32 P=256, params=0, operations=8192
Conv2d: S_c=3, F_in=32, F_out=8, P=256, params=1152, operations=882688
Batch norm: F_in=248 P=256, params=248, operations=190464
ReLU: F_in=248 P=256, params=0, operations=63488
Conv2d: S_c=1, F_in=248, F_out=32, P=256, params=3968, operations=3039232
Batch norm: F_in=32 P=256, params=32, operations=24576
ReLU: F_in=32 P=256, params=0, operations=8192
Conv2d: S_c=3, F_in=32, F_out=8, P=256, params=1152, operations=882688
Batch norm: F_in=256 P=256, params=256, operations=196608
ReLU: F_in=256 P=256, params=0, operations=65536
Conv2d: S_c=1, F_in=256, F_out=32, P=256, params=4096, operations=3137536
Batch norm: F_in=32 P=256, params=32, operations=24576
ReLU: F_in=32 P=256, params=0, operations=8192
Conv2d: S_c=3, F_in=32, F_out=8, P=256, params=1152, operations=882688
Batch norm: F_in=264 P=256, params=264, operations=202752
ReLU: F_in=264 P=256, params=0, operations=67584
Conv2d: S_c=1, F_in=264, F_out=32, P=256, params=4224, operations=3235840
Batch norm: F_in=32 P=256, params=32, operations=24576
ReLU: F_in=32 P=256, params=0, operations=8192
Conv2d: S_c=3, F_in=32, F_out=8, P=256, params=1152, operations=882688
Batch norm: F_in=272 P=256, params=272, operations=208896
ReLU: F_in=272 P=256, params=0, operations=69632
Conv2d: S_c=1, F_in=272, F_out=32, P=256, params=4352, operations=3334144
Batch norm: F_in=32 P=256, params=32, operations=24576
ReLU: F_in=32 P=256, params=0, operations=8192
Conv2d: S_c=3, F_in=32, F_out=8, P=256, params=1152, operations=882688
Batch norm: F_in=280 P=256, params=280, operations=215040
ReLU: F_in=280 P=256, params=0, operations=71680
Conv2d: S_c=1, F_in=280, F_out=32, P=256, params=4480, operations=3432448
Batch norm: F_in=32 P=256, params=32, operations=24576
ReLU: F_in=32 P=256, params=0, operations=8192
Conv2d: S_c=3, F_in=32, F_out=8, P=256, params=1152, operations=882688
Batch norm: F_in=288 P=256, params=288, operations=221184
ReLU: F_in=288 P=256, params=0, operations=73728
Conv2d: S_c=1, F_in=288, F_out=32, P=256, params=4608, operations=3530752
Batch norm: F_in=32 P=256, params=32, operations=24576
ReLU: F_in=32 P=256, params=0, operations=8192
Conv2d: S_c=3, F_in=32, F_out=8, P=256, params=1152, operations=882688
Batch norm: F_in=296 P=256, params=296, operations=227328
ReLU: F_in=296 P=256, params=0, operations=75776
Conv2d: S_c=1, F_in=296, F_out=32, P=256, params=4736, operations=3629056
Batch norm: F_in=32 P=256, params=32, operations=24576
ReLU: F_in=32 P=256, params=0, operations=8192
Conv2d: S_c=3, F_in=32, F_out=8, P=256, params=1152, operations=882688
Batch norm: F_in=304 P=256, params=304, operations=233472
ReLU: F_in=304 P=256, params=0, operations=77824
Conv2d: S_c=1, F_in=304, F_out=32, P=256, params=4864, operations=3727360
Batch norm: F_in=32 P=256, params=32, operations=24576
ReLU: F_in=32 P=256, params=0, operations=8192
Conv2d: S_c=3, F_in=32, F_out=8, P=256, params=1152, operations=882688
Batch norm: F_in=312 P=256, params=312, operations=239616
ReLU: F_in=312 P=256, params=0, operations=79872
Conv2d: S_c=1, F_in=312, F_out=32, P=256, params=4992, operations=3825664
Batch norm: F_in=32 P=256, params=32, operations=24576
ReLU: F_in=32 P=256, params=0, operations=8192
Conv2d: S_c=3, F_in=32, F_out=8, P=256, params=1152, operations=882688
Batch norm: F_in=320 P=256, params=320, operations=245760
ReLU: F_in=320 P=256, params=0, operations=81920
Conv2d: S_c=1, F_in=320, F_out=32, P=256, params=5120, operations=3923968
Batch norm: F_in=32 P=256, params=32, operations=24576
ReLU: F_in=32 P=256, params=0, operations=8192
Conv2d: S_c=3, F_in=32, F_out=8, P=256, params=1152, operations=882688
Batch norm: F_in=328 P=256, params=328, operations=251904
ReLU: F_in=328 P=256, params=0, operations=83968
Conv2d: S_c=1, F_in=328, F_out=32, P=256, params=5248, operations=4022272
Batch norm: F_in=32 P=256, params=32, operations=24576
ReLU: F_in=32 P=256, params=0, operations=8192
Conv2d: S_c=3, F_in=32, F_out=8, P=256, params=1152, operations=882688
Batch norm: F_in=336 P=256, params=336, operations=258048
ReLU: F_in=336 P=256, params=0, operations=86016
Conv2d: S_c=1, F_in=336, F_out=32, P=256, params=5376, operations=4120576
Batch norm: F_in=32 P=256, params=32, operations=24576
ReLU: F_in=32 P=256, params=0, operations=8192
Conv2d: S_c=3, F_in=32, F_out=8, P=256, params=1152, operations=882688
Batch norm: F_in=344 P=256, params=344, operations=264192
ReLU: F_in=344 P=256, params=0, operations=88064
Conv2d: S_c=1, F_in=344, F_out=32, P=256, params=5504, operations=4218880
Batch norm: F_in=32 P=256, params=32, operations=24576
ReLU: F_in=32 P=256, params=0, operations=8192
Conv2d: S_c=3, F_in=32, F_out=8, P=256, params=1152, operations=882688
Batch norm: F_in=352 P=256, params=352, operations=270336
ReLU: F_in=352 P=256, params=0, operations=90112
Conv2d: S_c=1, F_in=352, F_out=32, P=256, params=5632, operations=4317184
Batch norm: F_in=32 P=256, params=32, operations=24576
ReLU: F_in=32 P=256, params=0, operations=8192
Conv2d: S_c=3, F_in=32, F_out=8, P=256, params=1152, operations=882688
Batch norm: F_in=360 P=256, params=360, operations=276480
ReLU: F_in=360 P=256, params=0, operations=92160
Conv2d: S_c=1, F_in=360, F_out=32, P=256, params=5760, operations=4415488
Batch norm: F_in=32 P=256, params=32, operations=24576
ReLU: F_in=32 P=256, params=0, operations=8192
Conv2d: S_c=3, F_in=32, F_out=8, P=256, params=1152, operations=882688
Batch norm: F_in=368 P=256, params=368, operations=282624
ReLU: F_in=368 P=256, params=0, operations=94208
Conv2d: S_c=1, F_in=368, F_out=32, P=256, params=5888, operations=4513792
Batch norm: F_in=32 P=256, params=32, operations=24576
ReLU: F_in=32 P=256, params=0, operations=8192
Conv2d: S_c=3, F_in=32, F_out=8, P=256, params=1152, operations=882688
Batch norm: F_in=376 P=256, params=376, operations=288768
ReLU: F_in=376 P=256, params=0, operations=96256
Conv2d: S_c=1, F_in=376, F_out=32, P=256, params=6016, operations=4612096
Batch norm: F_in=32 P=256, params=32, operations=24576
ReLU: F_in=32 P=256, params=0, operations=8192
Conv2d: S_c=3, F_in=32, F_out=8, P=256, params=1152, operations=882688
Batch norm: F_in=384 P=256, params=384, operations=294912
ReLU: F_in=384 P=256, params=0, operations=98304
Conv2d: S_c=1, F_in=384, F_out=32, P=256, params=6144, operations=4710400
Batch norm: F_in=32 P=256, params=32, operations=24576
ReLU: F_in=32 P=256, params=0, operations=8192
Conv2d: S_c=3, F_in=32, F_out=8, P=256, params=1152, operations=882688
Batch norm: F_in=392 P=256, params=392, operations=301056
ReLU: F_in=392 P=256, params=0, operations=100352
Conv2d: S_c=1, F_in=392, F_out=196, P=256, params=38416, operations=29453312
AvgPool: S=2, F_in=196, P=256, params=0, operations=25088
Batch norm: F_in=196 P=64, params=196, operations=37632
ReLU: F_in=196 P=64, params=0, operations=12544
Conv2d: S_c=1, F_in=196, F_out=32, P=64, params=3136, operations=600064
Batch norm: F_in=32 P=64, params=32, operations=6144
ReLU: F_in=32 P=64, params=0, operations=2048
Conv2d: S_c=3, F_in=32, F_out=8, P=64, params=1152, operations=220672
Batch norm: F_in=204 P=64, params=204, operations=39168
ReLU: F_in=204 P=64, params=0, operations=13056
Conv2d: S_c=1, F_in=204, F_out=32, P=64, params=3264, operations=624640
Batch norm: F_in=32 P=64, params=32, operations=6144
ReLU: F_in=32 P=64, params=0, operations=2048
Conv2d: S_c=3, F_in=32, F_out=8, P=64, params=1152, operations=220672
Batch norm: F_in=212 P=64, params=212, operations=40704
ReLU: F_in=212 P=64, params=0, operations=13568
Conv2d: S_c=1, F_in=212, F_out=32, P=64, params=3392, operations=649216
Batch norm: F_in=32 P=64, params=32, operations=6144
ReLU: F_in=32 P=64, params=0, operations=2048
Conv2d: S_c=3, F_in=32, F_out=8, P=64, params=1152, operations=220672
Batch norm: F_in=220 P=64, params=220, operations=42240
ReLU: F_in=220 P=64, params=0, operations=14080
Conv2d: S_c=1, F_in=220, F_out=32, P=64, params=3520, operations=673792
Batch norm: F_in=32 P=64, params=32, operations=6144
ReLU: F_in=32 P=64, params=0, operations=2048
Conv2d: S_c=3, F_in=32, F_out=8, P=64, params=1152, operations=220672
Batch norm: F_in=228 P=64, params=228, operations=43776
ReLU: F_in=228 P=64, params=0, operations=14592
Conv2d: S_c=1, F_in=228, F_out=32, P=64, params=3648, operations=698368
Batch norm: F_in=32 P=64, params=32, operations=6144
ReLU: F_in=32 P=64, params=0, operations=2048
Conv2d: S_c=3, F_in=32, F_out=8, P=64, params=1152, operations=220672
Batch norm: F_in=236 P=64, params=236, operations=45312
ReLU: F_in=236 P=64, params=0, operations=15104
Conv2d: S_c=1, F_in=236, F_out=32, P=64, params=3776, operations=722944
Batch norm: F_in=32 P=64, params=32, operations=6144
ReLU: F_in=32 P=64, params=0, operations=2048
Conv2d: S_c=3, F_in=32, F_out=8, P=64, params=1152, operations=220672
Batch norm: F_in=244 P=64, params=244, operations=46848
ReLU: F_in=244 P=64, params=0, operations=15616
Conv2d: S_c=1, F_in=244, F_out=32, P=64, params=3904, operations=747520
Batch norm: F_in=32 P=64, params=32, operations=6144
ReLU: F_in=32 P=64, params=0, operations=2048
Conv2d: S_c=3, F_in=32, F_out=8, P=64, params=1152, operations=220672
Batch norm: F_in=252 P=64, params=252, operations=48384
ReLU: F_in=252 P=64, params=0, operations=16128
Conv2d: S_c=1, F_in=252, F_out=32, P=64, params=4032, operations=772096
Batch norm: F_in=32 P=64, params=32, operations=6144
ReLU: F_in=32 P=64, params=0, operations=2048
Conv2d: S_c=3, F_in=32, F_out=8, P=64, params=1152, operations=220672
Batch norm: F_in=260 P=64, params=260, operations=49920
ReLU: F_in=260 P=64, params=0, operations=16640
Conv2d: S_c=1, F_in=260, F_out=32, P=64, params=4160, operations=796672
Batch norm: F_in=32 P=64, params=32, operations=6144
ReLU: F_in=32 P=64, params=0, operations=2048
Conv2d: S_c=3, F_in=32, F_out=8, P=64, params=1152, operations=220672
Batch norm: F_in=268 P=64, params=268, operations=51456
ReLU: F_in=268 P=64, params=0, operations=17152
Conv2d: S_c=1, F_in=268, F_out=32, P=64, params=4288, operations=821248
Batch norm: F_in=32 P=64, params=32, operations=6144
ReLU: F_in=32 P=64, params=0, operations=2048
Conv2d: S_c=3, F_in=32, F_out=8, P=64, params=1152, operations=220672
Batch norm: F_in=276 P=64, params=276, operations=52992
ReLU: F_in=276 P=64, params=0, operations=17664
Conv2d: S_c=1, F_in=276, F_out=32, P=64, params=4416, operations=845824
Batch norm: F_in=32 P=64, params=32, operations=6144
ReLU: F_in=32 P=64, params=0, operations=2048
Conv2d: S_c=3, F_in=32, F_out=8, P=64, params=1152, operations=220672
Batch norm: F_in=284 P=64, params=284, operations=54528
ReLU: F_in=284 P=64, params=0, operations=18176
Conv2d: S_c=1, F_in=284, F_out=32, P=64, params=4544, operations=870400
Batch norm: F_in=32 P=64, params=32, operations=6144
ReLU: F_in=32 P=64, params=0, operations=2048
Conv2d: S_c=3, F_in=32, F_out=8, P=64, params=1152, operations=220672
Batch norm: F_in=292 P=64, params=292, operations=56064
ReLU: F_in=292 P=64, params=0, operations=18688
Conv2d: S_c=1, F_in=292, F_out=32, P=64, params=4672, operations=894976
Batch norm: F_in=32 P=64, params=32, operations=6144
ReLU: F_in=32 P=64, params=0, operations=2048
Conv2d: S_c=3, F_in=32, F_out=8, P=64, params=1152, operations=220672
Batch norm: F_in=300 P=64, params=300, operations=57600
ReLU: F_in=300 P=64, params=0, operations=19200
Conv2d: S_c=1, F_in=300, F_out=32, P=64, params=4800, operations=919552
Batch norm: F_in=32 P=64, params=32, operations=6144
ReLU: F_in=32 P=64, params=0, operations=2048
Conv2d: S_c=3, F_in=32, F_out=8, P=64, params=1152, operations=220672
Batch norm: F_in=308 P=64, params=308, operations=59136
ReLU: F_in=308 P=64, params=0, operations=19712
Conv2d: S_c=1, F_in=308, F_out=32, P=64, params=4928, operations=944128
Batch norm: F_in=32 P=64, params=32, operations=6144
ReLU: F_in=32 P=64, params=0, operations=2048
Conv2d: S_c=3, F_in=32, F_out=8, P=64, params=1152, operations=220672
Batch norm: F_in=316 P=64, params=316, operations=60672
ReLU: F_in=316 P=64, params=0, operations=20224
Conv2d: S_c=1, F_in=316, F_out=32, P=64, params=5056, operations=968704
Batch norm: F_in=32 P=64, params=32, operations=6144
ReLU: F_in=32 P=64, params=0, operations=2048
Conv2d: S_c=3, F_in=32, F_out=8, P=64, params=1152, operations=220672
Batch norm: F_in=324 P=64, params=324, operations=62208
ReLU: F_in=324 P=64, params=0, operations=20736
Conv2d: S_c=1, F_in=324, F_out=32, P=64, params=5184, operations=993280
Batch norm: F_in=32 P=64, params=32, operations=6144
ReLU: F_in=32 P=64, params=0, operations=2048
Conv2d: S_c=3, F_in=32, F_out=8, P=64, params=1152, operations=220672
Batch norm: F_in=332 P=64, params=332, operations=63744
ReLU: F_in=332 P=64, params=0, operations=21248
Conv2d: S_c=1, F_in=332, F_out=32, P=64, params=5312, operations=1017856
Batch norm: F_in=32 P=64, params=32, operations=6144
ReLU: F_in=32 P=64, params=0, operations=2048
Conv2d: S_c=3, F_in=32, F_out=8, P=64, params=1152, operations=220672
Batch norm: F_in=340 P=64, params=340, operations=65280
ReLU: F_in=340 P=64, params=0, operations=21760
Conv2d: S_c=1, F_in=340, F_out=32, P=64, params=5440, operations=1042432
Batch norm: F_in=32 P=64, params=32, operations=6144
ReLU: F_in=32 P=64, params=0, operations=2048
Conv2d: S_c=3, F_in=32, F_out=8, P=64, params=1152, operations=220672
Batch norm: F_in=348 P=64, params=348, operations=66816
ReLU: F_in=348 P=64, params=0, operations=22272
Conv2d: S_c=1, F_in=348, F_out=32, P=64, params=5568, operations=1067008
Batch norm: F_in=32 P=64, params=32, operations=6144
ReLU: F_in=32 P=64, params=0, operations=2048
Conv2d: S_c=3, F_in=32, F_out=8, P=64, params=1152, operations=220672
Batch norm: F_in=356 P=64, params=356, operations=68352
ReLU: F_in=356 P=64, params=0, operations=22784
Conv2d: S_c=1, F_in=356, F_out=32, P=64, params=5696, operations=1091584
Batch norm: F_in=32 P=64, params=32, operations=6144
ReLU: F_in=32 P=64, params=0, operations=2048
Conv2d: S_c=3, F_in=32, F_out=8, P=64, params=1152, operations=220672
Batch norm: F_in=364 P=64, params=364, operations=69888
ReLU: F_in=364 P=64, params=0, operations=23296
Conv2d: S_c=1, F_in=364, F_out=32, P=64, params=5824, operations=1116160
Batch norm: F_in=32 P=64, params=32, operations=6144
ReLU: F_in=32 P=64, params=0, operations=2048
Conv2d: S_c=3, F_in=32, F_out=8, P=64, params=1152, operations=220672
Batch norm: F_in=372 P=64, params=372, operations=71424
ReLU: F_in=372 P=64, params=0, operations=23808
Conv2d: S_c=1, F_in=372, F_out=32, P=64, params=5952, operations=1140736
Batch norm: F_in=32 P=64, params=32, operations=6144
ReLU: F_in=32 P=64, params=0, operations=2048
Conv2d: S_c=3, F_in=32, F_out=8, P=64, params=1152, operations=220672
Batch norm: F_in=380 P=64, params=380, operations=72960
ReLU: F_in=380 P=64, params=0, operations=24320
Conv2d: S_c=1, F_in=380, F_out=32, P=64, params=6080, operations=1165312
Batch norm: F_in=32 P=64, params=32, operations=6144
ReLU: F_in=32 P=64, params=0, operations=2048
Conv2d: S_c=3, F_in=32, F_out=8, P=64, params=1152, operations=220672
Batch norm: F_in=388 P=64, params=388, operations=74496
ReLU: F_in=388 P=64, params=0, operations=24832
Conv2d: S_c=1, F_in=388, F_out=32, P=64, params=6208, operations=1189888
Batch norm: F_in=32 P=64, params=32, operations=6144
ReLU: F_in=32 P=64, params=0, operations=2048
Conv2d: S_c=3, F_in=32, F_out=8, P=64, params=1152, operations=220672
Batch norm: F_in=396 P=64, params=396, operations=76032
ReLU: F_in=396 P=64, params=0, operations=25344
Conv2d: S_c=1, F_in=396, F_out=32, P=64, params=6336, operations=1214464
Batch norm: F_in=32 P=64, params=32, operations=6144
ReLU: F_in=32 P=64, params=0, operations=2048
Conv2d: S_c=3, F_in=32, F_out=8, P=64, params=1152, operations=220672
Batch norm: F_in=404 P=64, params=404, operations=77568
ReLU: F_in=404 P=64, params=0, operations=25856
Conv2d: S_c=1, F_in=404, F_out=32, P=64, params=6464, operations=1239040
Batch norm: F_in=32 P=64, params=32, operations=6144
ReLU: F_in=32 P=64, params=0, operations=2048
Conv2d: S_c=3, F_in=32, F_out=8, P=64, params=1152, operations=220672
Batch norm: F_in=412 P=64, params=412, operations=79104
ReLU: F_in=412 P=64, params=0, operations=26368
Conv2d: S_c=1, F_in=412, F_out=32, P=64, params=6592, operations=1263616
Batch norm: F_in=32 P=64, params=32, operations=6144
ReLU: F_in=32 P=64, params=0, operations=2048
Conv2d: S_c=3, F_in=32, F_out=8, P=64, params=1152, operations=220672
Batch norm: F_in=420 P=64, params=420, operations=80640
ReLU: F_in=420 P=64, params=0, operations=26880
Conv2d: S_c=1, F_in=420, F_out=32, P=64, params=6720, operations=1288192
Batch norm: F_in=32 P=64, params=32, operations=6144
ReLU: F_in=32 P=64, params=0, operations=2048
Conv2d: S_c=3, F_in=32, F_out=8, P=64, params=1152, operations=220672
Batch norm: F_in=428 P=64, params=428, operations=82176
ReLU: F_in=428 P=64, params=0, operations=27392
Conv2d: S_c=1, F_in=428, F_out=32, P=64, params=6848, operations=1312768
Batch norm: F_in=32 P=64, params=32, operations=6144
ReLU: F_in=32 P=64, params=0, operations=2048
Conv2d: S_c=3, F_in=32, F_out=8, P=64, params=1152, operations=220672
Batch norm: F_in=436 P=64, params=436, operations=83712
ReLU: F_in=436 P=64, params=0, operations=27904
Conv2d: S_c=1, F_in=436, F_out=32, P=64, params=6976, operations=1337344
Batch norm: F_in=32 P=64, params=32, operations=6144
ReLU: F_in=32 P=64, params=0, operations=2048
Conv2d: S_c=3, F_in=32, F_out=8, P=64, params=1152, operations=220672
Batch norm: F_in=444 P=64, params=444, operations=85248
ReLU: F_in=444 P=64, params=0, operations=28416
Conv2d: S_c=1, F_in=444, F_out=32, P=64, params=7104, operations=1361920
Batch norm: F_in=32 P=64, params=32, operations=6144
ReLU: F_in=32 P=64, params=0, operations=2048
Conv2d: S_c=3, F_in=32, F_out=8, P=64, params=1152, operations=220672
Batch norm: F_in=452 P=64, params=452, operations=86784
ReLU: F_in=452 P=64, params=0, operations=28928
Conv2d: S_c=1, F_in=452, F_out=226, P=64, params=51076, operations=9792128
AvgPool: S=2, F_in=226, P=64, params=0, operations=7232
Batch norm: F_in=226 P=16, params=226, operations=10848
ReLU: F_in=226 P=16, params=0, operations=3616
AvgPool: S=4, F_in=226, P=16, params=0, operations=904
Linear: F_in=226, F_out=100, params=11350, operations=33800

\end{Verbatim}


\end{document} 
